\documentclass{article}[jsarticle]
\usepackage[T1]{fontenc}
\usepackage[dvipdfmx]{hyperref}
\usepackage{lmodern}
\usepackage{latexsym}
\usepackage{amsfonts}
\usepackage{amssymb}
\usepackage{mathtools}
\usepackage{nccmath}
\usepackage{amsthm}
\usepackage{multirow}
\usepackage[dvipdfmx]{graphicx}
\usepackage{wrapfig}
\usepackage{here}
\usepackage{float}
\usepackage{ascmac}
\usepackage{url}

% Generated by ChatGPT
\usepackage{listings}
\usepackage{xcolor}

\lstset{
    basicstyle=\ttfamily\color{white},
    numbers=none,  % Line numbers
    numberstyle=\tiny\color{white},
    numbersep=5pt,
    tabsize=2,
    extendedchars=true,
    breaklines=true,
    keywordstyle=\color[rgb]{0.58,0.00,0.83},
    stringstyle=\color[rgb]{0.81,0.36,0.00},
    identifierstyle=\color{white},
    commentstyle=\color[rgb]{0.34,0.62,0.16},
    rulecolor=\color[rgb]{0.5,0.5,0.5},
    xleftmargin=0.1cm,    % Left margin
    xrightmargin=0.1cm,   % Right margin
    language=python,
    backgroundcolor=\color[rgb]{0.13,0.13,0.13},
    showspaces=false,
    showstringspaces=false
}



\title{2023年度 特別研究2 研究状況報告書}
\author{高林秀 \\ 三宅研究室 博士前期課程1年 \\ V-CampusID : 23vr008n}
\date{\today}

\begin{document}

\maketitle

\begin{abstract}
    本稿は本年度特別研究2の研究状況報告を記載するものである.\par
    本稿の構成は,第一章において研究計画の概要,第二章において研究計画が確定した昨年度7月から本年1月までに行った研究活動について,
    第三章において現在の研究の進捗状況と課題点について,第四章において来年度特別研究3以降の修士論文までの計画を示す.\par 
    %TODO:第一章の概要
    %TODO:第二章の概要
    %TODO:第三章の概要
    %TODO:第四章の概要
    なお巻末には参考文献と,本稿記載の実験環境を格納したリポジトリのURLを記載する.
\end{abstract}



\tableofcontents


\section{研究計画の概要}
本章では,昨年の特別研究1で提出した研究計画について,研究テーマ,研究背景,本研究が目指す目標について簡単に記載する.
詳細は,巻末に添付する特別研究1の研究計画書を参照されたい.
\subsection{研究テーマ}
    \centerline{
        \textbf{マルチエージェント強化学習を利用した,} \\
        \textbf{自律型ドローンによる災害時の援助物資輸送アプローチ}
    }

    大規模災害時の被災者救助と物資輸送の効率化を目指す.
    具体的には,ドローンとマルチエージェント強化学習を組み合わせた新たなアプローチを開発し,その有効性を検証する.
\subsection{研究背景}
\paragraph{自衛隊員の不足} \par 

災害大国である我が国において,被災者の捜索,被害状況の把握,
救援物資の現地輸送といった対応は,迅速かつ効率的に行われなければならない.しかし近年,そのような災害対応を一任務とする,
自衛隊員の人材が不足している,あるいは今後不足する事態が予想されるなど,好ましくない状況が続いている.
\paragraph{レベル4飛行の解禁} \par

レベル4飛行とは,有人地帯での目視外飛行のこと.目視で監視できない状態で,有人地帯上空を自律飛行することができる飛行レベルを指す.
我が国では2022年12月5日に改正航空法が施行され,レベル4飛行が解禁された.\par 
これにより,現在ドローンの災害対応における有効性に注目が集まっている.

\subsection{本研究が目指す目標}
本研究では,災害で孤立した街や都市を再現し,複数のドローンをエージェントとした強化学習に
よって,被災者の捜索,救援物資の輸送の最適化を模索し,自律型ドローンの災害時における有用性を
検証する.\par 
また,学習したモデルを用いて,実際のドローンを用いた飛行までの技術開発と飛行実験まで行うことを目標とする.

\section{研究活動報告}
本章では昨年度7月の研究計画の確定から本年1月までに行った研究活動について記載する.
\subsection{災害対応における問題点の調査}
\subsection{ドローンの災害時活用事例調査}
\subsection{ドローンの性能限界と法的な問題点の調査}
\subsection{シングルエージェントでの簡単な物資輸送シミュレーション}
\subsection{マルチエージェントでの簡単な物資輸送シミュレーション}
\subsection{強化学習モデルと実機ドローンの連携を行うための技術調査}

\section{研究の進捗状況と課題点}

\section{来年度以降の計画}

\section{参考文献}

\section{実験環境}

\end{document}